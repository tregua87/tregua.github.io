%%%%%%%%%%%%%%%%%%%%%%%%%%%%%%%%%%%%%%%%% 
% Medium Length Professional CV
% LaTeX Template
% Version 2.0 (8/5/13)
%
% This template has been downloaded from:
% http://www.LaTeXTemplates.com
%
% Original author:
% Trey Hunner (http://www.treyhunner.com/)
%
% Important note:
% This template requires the resume.cls file to be in the same directory as the
% .tex file. The resume.cls file provides the resume style used for structuring the
% document.
%
%%%%%%%%%%%%%%%%%%%%%%%%%%%%%%%%%%%%%%%%%

%----------------------------------------------------------------------------------------
%	PACKAGES AND OTHER DOCUMENT CONFIGURATIONS
%----------------------------------------------------------------------------------------

\documentclass{resume} % Use the custom resume.cls style

\usepackage[left=0.75in,top=0.6in,right=0.75in,bottom=0.6in]{geometry} % Document margins

\usepackage{enumitem}

\name{Flavio Toffalini} % Your name
%\address{XXX \\ Lausanne, Switzerland, } % Your address
%\address{123 Pleasant Lane \\ City, State 12345} % Your secondary =addess (optional)
\address{(+41)~$\cdot$~77 239 72 25 \\ flavio.toffalini@epfl.ch} % 
%Your phone number and email

\begin{document}
	
My research interest covers many aspects of system security. My Ph.D. background
focuses on software security for Trusted Execution Environment. In my current position, I am intensively working on automatic testing and mitigation applied
to many system levels, from user-space to virtual devices.

\begin{rSection}{Current Position: PostDoc in the HexHive laboratory at EPFL}
	
	{\bf \'Ecole Polytechnique F\'ed\'erale de Lausanne (EPFL), Switzerland} 
	\hfill {\em Nov 2021 to Now} \\
	PostDoc, supervised by Prof. Mathias Payer \\
	Topic: fuzzing, mitigation, software analysis

\end{rSection}

%----------------------------------------------------------------------------------------
%	EDUCATION SECTION
%----------------------------------------------------------------------------------------

\begin{rSection}{Education}
    
{\bf Singapore University of Technology and Design, Singapore} \hfill {\em Jan
2017 - Sep 2021} \\
Ph.D., supervisor Prof. Jianying Zhou \\
Topic: trusted computing, system security \\
Thesis Title: Challenges, threats, and novel defenses for Trusted Execution
Environments

{\bf University of Verona, Italy} \hfill {\em Sep 2012 - Oct 2015} \\
M.S. in Computer Science and Engineering 108$/$110, GPA 3,9/4 \\
Supervisor Prof. Damiano Carra \\
Master thesis topic: Google dorks, Web security


{\bf University of Pavia, Italy} \hfill {\em Sep 2007 - Dec 2009} \\ 
B.S. in Computer Engineer 101$/$110, GPA 3,67/4 \\
Supervisor Prof. Paolo Gamba \\


\end{rSection}

%----------------------------------------------------------------------------------------
%	WORK EXPERIENCE SECTION
%----------------------------------------------------------------------------------------

\begin{rSection}{Academic Activities}

\begin{rSubsection}{King's College London}{Nov 2019 - Mar 2020}{Visiting 
fellow, supervised by Prof. Lorenzo Cavallaro}{London, UK}
	\item Topic: trusted computing, system security
\end{rSubsection}

\begin{rSubsection}{University of Padua}{Jan 2018 - Aug 2018}{Visiting fellow, supervised by Prof. Mauro Conti}{Padua, Italy}
    \item Topic: trusted computing, system security
\end{rSubsection}

% \begin{rSubsection}{Singapore University of Technology and Design}{Aug 2016 - Dec 2017}{Research Assistant, supervised by Prof. Mart\'in Ochoa}{Singapore, Singapore}
%     \item Topic: insider threats, trusted computing
% \end{rSubsection}

\begin{rSubsection}{University of Verona}{Dec 2015 - July 2016}{Research Assistant, supervised by Prof. Fausto Spoto}{Verona, Italy}
\item Topic: static analysis of Android applications
\end{rSubsection}

\begin{rSubsection}{Eurecom}{April 2015 - July 2015}{Visiting fellow, supervised by Prof. Davide Balzarotti}{Biot, France}
\item Topic: Google dorks, Web security
\end{rSubsection}

\end{rSection}
\newpage
\begin{rSection}{Selected Publications}

% \begin{rSeparator}{2023}
% \end{rSeparator}

% \begin{rSubsection}{Crystallizer: A Hybrid Path Analysis Framework To Aid in
% 	Uncovering Deserialization Vulnerabilities}{major reviion -
% 	2023}{Major revision at Foundations of Software Engineering (FSE)}{}
% 	\item Srivastava P., \textbf{Toffalini F.}, Vorobyov K., Gauthier F.,
% 	Bianchi A, and Payer M.
% \end{rSubsection}


\textbf{Conference}\begin{enumerate}[label={[C\arabic*]},leftmargin=5mm]
\item Zheng H., Zhang J., Huang Y., Ren Z., Wang H., Cao C., Zhang Y., \textbf{Toffalini F.}, Payer M.\\``FishFuzz: Throwing Larger Nets to Catch Deeper Bugs'' Proceeding of the 32nd USENIX Security Symposium (Usenix SEC 2023)
\item Xu J., Di Bartolomeo L., \textbf{Toffalini F.}, Mao B., Payer M.\\``WarpAttack: Bypassing CFI through Compiler-Introduced Double-Fetches'' Proceeding of the 44th IEEE Symposium on Security and Privacy (S\&P 2023)
\item Liu Q., \textbf{Toffalini F.}, Zhou Y., Payer M.\\``ViDeZZO: Dependency-aware Virtual Device Fuzzing'' Proceeding of the 44th IEEE Symposium on Security and Privacy (S\&P 2023)
\item \textbf{Toffalini F.}, Payer M., Zhou J., Cavallaro L.\\``Designing a Provenance Analysis for SGX Enclaves'' Proceeding of the 38th Annual Computer Security Applications Conference (ACSAC 2022)
\item Jiang Z., Gan S., Herrera A., \textbf{Toffalini F.}, Romerio L., Tang C., Egele M., Zhang C., Payer M.\\``Evocatio: Conjuring Bug Capabilities from a Single PoC'' Proceeding of the ACM SIGSAC Conference on Computer and Communications Security (CCS 2022)
\item \textbf{Toffalini F.}, Graziano M., Conti M., Zhou J.\\``SnakeGX: a sneaky attack against SGX Enclaves'' Proceeding of the 19th International Conference on Applied Cryptography and Network Security (ACNS 2022)
\item \textbf{Toffalini F.}, Losiouk E., Biondo A., Zhou J., Conti M.\\``ScaRR: Scalable Runtime Remote Attestation for Complex Systems'' Proceeding of the 22nd International Symposium on Research in Attacks, Intrusions and Defenses (RAID 2019)
\item \textbf{Toffalini F.}, Ochoa M., Sun J., Zhou J.\\``Careful-Packing: A Practical and Scalable Anti-Tampering Software Protection enforced by Trusted Computing'' Proceeding of the 9th ACM Conference on Data and Application Security and Privacy (CODASPY 2019)
\item \textbf{Toffalini F.}, Sun J., Ochoa M.\\``Static Analysis of Context Leaks in Android Applications'' Proceeding of the 40th International Conference on Software Engineering: Software Engineering in Practice (SEPA@ICSE)
\item \textbf{Toffalini F.}, Abba' M., Carra D., Balzarotti D.\\``Google Dorks: Analysis, Creation, and new Defenses'' Proceeding of the 13th International Conference of Detection of Intrusions, Malware, and Vulnerability Assessment, (DIMVA 2016)
\end{enumerate}
\textbf{Workshop}\begin{enumerate}[label={[W\arabic*]},leftmargin=5mm]
\item \textbf{Toffalini F.}, Homoliak I., Harilal A., Binder A., Ochoa M.\\``Detection of Masqueraders Based on Graph Partitioning of File System Access Events'' Proceeding of IEEE Security and Privacy Workshops (SPW)
\item Harilal A., \textbf{Toffalini F.}, John C., Guarnizo J., Homoliak I., Ochoa M.\\``TWOS: A Dataset of Malicious Insider Threat Behavior Based on Gamified Competition'' Proceeding of the 9th ACM CCS International Workshop on Managing Insider Security Threats (MIST)
\item De Stefani F., Gamba P., Goldoni E., Savioli A., Silvestri D., \textbf{Toffalini F.}\\``REnvDB, a RESTful Database for Pervasive Environmental Wireless Sensor Networks'' Proceeding of the 30th IEEE International Conference on Distributed Computing Systems Workshops
\end{enumerate}
\textbf{Journal}\begin{enumerate}[label={[J\arabic*]},leftmargin=5mm]
\item \textbf{Toffalini F.}, Oliveri A., Graziano M., Zhou J., Balzarotti D.\\``The evidence beyond the wall: Memory forensics in SGX environments'' Forensic Science International: Digital Investigation, 2021
\item Homoliak I., \textbf{Toffalini F.}, Guarnizo J., Elovici Y., Ochoa M.\\``Insight Into Insiders and IT: A Survey of Insider Threat Taxonomies, Analysis, Modeling, and Countermeasures'' ACM Computing Surveys (CSUR), 2019
\item \textbf{Toffalini F.}, Sun J., Ochoa M.\\``Practical static analysis of context leaks in Android applications'' Software: Practice and Experience, 2019
\item Harilal A., \textbf{Toffalini F.}, Homoliak I., John C., Guarnizo J., Mondal S., Ochoa M.\\``The Wolf Of SUTD (TWOS): A Dataset of Malicious Insider Threat Behavior Based on a Gamified Competition'' Journal of Wireless Mobile Networks, Ubiquitous Computing, and Dependable Applications (JoWUA), 2018
\end{enumerate}


\textbf{Conference -- old}

\begin{enumerate}[label={[C\arabic*]},leftmargin=5mm]
	% \item Zheng H., Zhang J., Huang Y., Ren Z., Wang H., Cao C., Zhang Y.,
	% \textbf{Toffalini F.}, and Payer M.\\``FishFuzz: Throwing Larger Nets to
	% Catch Deeper Bugs'' Proceeding of the 32nd USENIX Security Symposium (Usenix
	% SEC 2023)
	% \item Xu J., Di Bartolomeo L., \textbf{Toffalini F.}, Mao B. and Payer
	% M.\\``WarpAttack: Bypassing CFI through Compiler-Introduced Double-Fetches''
	% Proceedings of the 44th IEEE Symposium on Security and Privacy (S\&P 2023)
	% \item Liu Q., \textbf{Toffalini F.}, Zhou Y., and Payer M.\\``ViDeZZO:
	% Dependency-aware Virtual Device Fuzzing'' Proceedings of the 44th IEEE
	% Symposium on Security and Privacy (S\&P 2023)
	\item \textbf{Toffalini F.}, Payer M., Zhou J., and Cavallaro
	L.\\``Designing a Provenance Analysis for SGX Enclaves'' Proceedings of the
	38th Annual Computer Security Applications Conference (ACSAC 2022)
	\item \textbf{Toffalini F.}, Graziano M., Conti M., and Zhou J.\\``SnakeGX:
a sneaky attack against SGX Enclaves'' Proceedings of the 19th International
Conference on Applied Cryptography and Network Security (ACNS 2022)
	\item \textbf{Toffalini F.}, Losiouk E., Biondo A., Zhou J., Conti
	M.\\``ScaRR: Scalable Runtime Remote Attestation for Complex Systems''
	Proceedings of the 22nd International Symposium on Research in Attacks,
	Intrusions and Defenses (RAID 2019)
	\item\textbf{Toffalini F.}, Ochoa M., Sun J., and Zhou J.\\ ``Careful-Packing:
	A Practical and Scalable Anti-Tampering Software Protection enforced by
	Trusted Computing'' Proceedings of the 9th ACM Conference on Data and
	Application Security and Privacy (CODASPY 2019)
\end{enumerate}

% \begin{rSubsection}{HolA: Holistic and Autonomous Attestation for IoT Networks}{workshop - 2022}{Applied Cryptography and Network Security Workshops (ACNS)}{}
% 	\item Visintin A., \textbf{Toffalini F.}, Losiouk E., Conti M., and Zhou J.
% \end{rSubsection} 


%\begin{rSubsection}{Practical static analysis of context leaks in Android 
%applications}{journal - 2019}{Software: Practice and Experience}{}
%    \item \textbf{Toffalini F.}, Sun J., and Ochoa M.
%\end{rSubsection}        

% \begin{rSubsection}{Static Analysis of Context Leaks in Android Applications}{conference - 2018}{The 40th International Conference on Software Engineering:\\Software Engineering in Practice (SEPA@ICSE)}{}
%     \item \textbf{Toffalini F.}, Sun J, and Ochoa M.
% \end{rSubsection}

% \begin{rSubsection}{Detection of Masqueraders Based on Graph Partitioning\\of File System Access Events}{workshop - 2018}{IEEE Security and Privacy Workshops (SPW)}{}
%     \item \textbf{Toffalini F.}, Homoliak I., Harilal A., Binder A., and Ochoa M.
% \end{rSubsection}


% \begin{rSubsection}{The Wolf Of SUTD (TWOS): A Dataset of Malicious Insider 
% Threat\\Behavior Based on a Gamified Competition}{journal - 2018}{Journal of 
% Wireless Mobile Networks, Ubiquitous Computing, and\\Dependable Applications 
% (JoWUA)}{}
%    \item Harilal A., \textbf{Toffalini F.}, Homoliak I., Castellanos J., 
% Guarnizo J., Mondal S., and Ochoa M.
% \end{rSubsection}


% \begin{rSubsection}{TWOS: A Dataset of Malicious Insider Threat Behavior Based on a\\Gamified Competition}{workshop - 2017}{The 9th ACM CCS International Workshop on Managing Insider Security Threats (MIST)}{}
% \item Harilal A., \textbf{Toffalini F.}, Castellanos J., Guarnizo J., Homoliak I., and Ochoa M.
% \end{rSubsection}

% \begin{rSubsection}{Google Dorks: Analysis, Creation, and new Defenses}{conference - 2016}
% {Detection of Intrusions and Malware, and Vulnerability Assessment\\The 13th International Conference, (DIMVA)}{}
%     \item \textbf{Toffalini F.}, Abba'  M., Carra D. and Balzarotti D.
% \end{rSubsection}

% \begin{rSubsection}{REnvDB, a RESTful Database for Pervasive Environmental\\ 
% Wireless Sensor Networks}{workshop - 2010}{IEEE Distributed Computing Systems 
% Workshops (ICDCSW)}{}
%    \item De Stefani F., Gamba P., Goldoni E., Savioli A., Silvestri D., and 
% \textbf{Toffalini F.}
% \end{rSubsection}
    
\end{rSection}

%----------------------------------------------------------------------------------------
%	TECHNICAL STRENGTHS SECTION
%----------------------------------------------------------------------------------------

\begin{rSection}{Academic Service}

\begin{tabular}{ @{} >{\bfseries}l @{\hspace{6ex}} l }
ISSTA reviewer 2024 \\
NDSS reviewer 2022/23/24 \\
DIMVA reviewer 2022/23 \\
Usenix SEC AE reviewer 2022 \\
EuroSP shadow-reviewer 2020 \\
TIFS reviewer 2018/19 \\
\end{tabular}

\end{rSection}

%----------------------------------------------------------------------------------------
%	EXAMPLE SECTION
%----------------------------------------------------------------------------------------

%\begin{rSection}{Section Name}

%Section content\ldots

%\end{rSection}

%----------------------------------------------------------------------------------------

\end{document}
